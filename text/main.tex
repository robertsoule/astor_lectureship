\documentclass[11pt]{article}

\usepackage{graphicx}

% Margins
\topmargin=-0.45in
\evensidemargin=0in
\oddsidemargin=0in
\textwidth=6.5in
\textheight=9.0in
\headsep=0.25in


\title{ Robert Soul\'{e} Career Summary}
\author{  }
\date{ }

\begin{document}

\maketitle	
\thispagestyle{empty}


Robert Soul\'{e} is a Tenured Associate Professor in the Department of
Computer Science at Yale University, with a joint appointment in the
Department of Electrical Engineering. He is a Senior Member of the ACM
and IEEE.  Prior to joining Yale, he was an Associate Professor in the
Faculty of Informatics at the Universit\`{a} della Svizzera italiana
in Lugano, Switzerland.

Professor Soul\'{e} is a leading computer networking researcher,
well-known in the research community for his work on software-defined
networks (SDN). He is among the top-five most cited authors from the
Symposium of SDN Research (SOSR) conference series. He is particularly
recognized for his foundational contributions to in-network
computing. His seminal work on this topic has been recognized with
Best Paper awards at USENIX NSDI 2018, and ACM CoNEXT 2020, as well as
the 2019 ACM SIGMETRICS Highlights Beyond SIGMETRICS.


Professor Soul\'{e} made important contributions to the development of
the P4 language, the Barefoot Tofino Programmable Switch Chip, and the
development of applications that used those technologies. He led the
development team for Barefoot Networks (and later Intel)'s Deep
Insight Network telemetry system. He was a member of the P4 Technical
Steering Team and served as Chair of the P4 Education Working
Group. For his contributions to the Barefoot Tofino Programmable
Switch Chips and the P4 Programming Language, Professor Soul\'{e} was
a recipient of the 2025 ACM SIGCOMM Networking Systems Award.

His recent work has focused on the environmental impact of computer
networks and distributed systems. He is Co-PI on a joint UK
Engineering and Physical Sciences Research Council (EPSRC) / US
National Science Foundation (NSF) grant to study carbon-aware
networking and the recipient of a Yale University Planetary Solutions
Acceleration Award for developing an energy-proportional switch. He is
particularly interested in exploring the trade-offs in energy
efficiency and fault-tolerance in distributed systems.

During his Ph.D. studies, Professor Soul\'{e} explored large-scale,
distributed stream processing. He was a long-term co-op in the
Data Intensive Systems and Analytics at IBM T. J. Watson Research
Center, and his thesis work contributed to the SPL compiler for the
IBM Infosphere Streams stream processing engine. For this work,
Soul\'{e} received four IBM Invention Achievement Awards and an IBM
Invention Plateau Award. The Streams project was awarded an R\&D 100
Award For The Most Breakthrough Innovation.  His work on compiler
optimizations for streaming languages received the best paper award at
at ACM DEBS 2012.

Professor Soul\'{e} has authored more than 65 peer reviewed
publications and is an inventor on 11 US patents. He regularly serves
on technical program committees for top systems conferences, including
NSDI, SIGCOMM, OSDI, and EuroSys.

He has held consulting positions in industry at Barefoot Networks,
Intel Corporation, and most recently, Cerebras Systems, where he works
on the network architecture for their AI inference and training data
centers.  Prior to pursuing a career in academia, he worked at several
companies in industry, including Morgan Stanley, Bloomberg, and
LimeWire.

Professor Soul\'{e} received a Ph.D. from New York University and a
B.A. from Brown University. After completing his Ph.D., he was a
post-doctoral associate at Cornell University.








\end{document}
